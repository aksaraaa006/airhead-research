\documentclass[11pt]{article}
\usepackage{latexsym}

\title{Project proposal: Learning Games with Self Organizing Maps}

\author{Keith Stevens}


\begin{document}
\maketitle

\section{Introduction}
Recently a significant amount of research has gone into modeling how child
learners acquire language, taking into account the phenomenon seen, while also
attempting to mirror the environments in which language is acquired.  One of the
key features modeled in these environments is the presents of referential
uncertainty, where a single utterance is heard with multiple meanings appearing
simultaneously, and only one is correct.  Much of
this work has been done using statistical models, such as with bayesian
statistics, and a few models have been purely symbolic.  Unfortunately, little
work has been done which utilized neural learning methods in a learning
environment similar real situations.  A recent successful connectionist system
\cite{liDevLex} closely models learning phenomenon, but only works with a
carefully designed meaning representation that does not take into account
referential uncertainty.

This project proposal is essentially to utilize Self Organizing Maps, with most
technical details best outlined in
\cite{liDevLex} in a simple learning game to determine if multiple agents can
converge on the same language.

\section{The Learning Game}
The simple learning game for this project will consist of multiple agents, all
of which initially have no lexical knowledge.  In the world containing theses
agents, there will be a fixed number of objects, each with it's own unique
representation.  A simple representation for each object would be to represent
each object as a feature vector, with most of the features not being present,
and only a few having some positive real value.  
The agents will then place a series of learning games described as follows, and is best described in \cite{vogtLearningSim}: 

\begin{enumerate}
  \item Two unique agents will be selected.  One will be the speaker, one will
  be the hearer.
  \item The speaker will select one object from the environment to be the topic
  $T$, and $x$ others to be within the visible context $C$.
  \item The speaker selects the word which best represents $T$, and if no
  fitting word can be found, the speaker generates a new word.  A mapping of $T$
  to the chosen word is reinforced for the speaker.
  \item $T$, $C$, are then combined and sent to the hearer along with the chosen word (in a more advanced simulation the hearer would simply observe $T$ and $C$ on it's own).
  \item The hearer attempts to learn a mapping for the heard word and the
  present meaning.
\end{enumerate}

Although somewhat simplistic, this learning game exhibits the referential
uncertainty in word learning, which hasn't yet been tried with Self Organizing
Maps.

\section{Steps for Learning}
Each agent will internally have a set of Self Organizing Maps, each connected
with associative links, the basic equations and more advanced models are introduced in
\cite{liDevLex,DisLex}.  

Since learning under referential uncertainty might not be best realized with
SOMs, the project will work in stages.  A preliminary outline is:

\begin{enumerate}
  \item Use a single SOM to map observed meaning vectors to the heard utterance
  (where utterances initially have no extra context).  Call this SOM the M-SOM
  \item Include a second SOM mapping a word to a vector formatted phonological
  representation and connect to the original map via hebbian learning, call this
  SOM the P-SOM, \cite{liDevLex}.
  \item Introduce contexts in learning games
  \item Modify the M-SOM to handle the presence of additional context, possibly
  by adding another weight vector which scales the input giving favor to
  features which have stayed constant over several learning games, and reduces
  the impact of frequently changing features.
  \item Extend each SOM to slowly exhibit Adaptive Resonance theory, and the
  ability to grow as seen in \cite{liDevLex}.
  \item Begin introducing words representing multiple objects.
\end{enumerate}

\section{Proposed Evaluation}
Since the central question for this project is whether or not SOMS can handle
referential uncertainty, the first two stages should allow the agents to all
converge onto a language, so prior to advancing in the stages it should be
checked that the learning simulation presented converges using some fixed
strategy.

A simple measure of language convergence is to check
that for every object in the environment, all the agents agree on some word
through their internal mappings.  Of additional interest would also be to see
what kinds of cluster occurs within the agents, especially withing the M-SOMs.
This could best be evaluated by generating objects in a systematic fashion such
that there are general categories of objects which have similar, but not exact,
representations.  If all the agents then categorize the objects in a similar
fashion, then the learning simulation could also be considered a success.

\bibliographystyle{IEEEtran}
\bibliography{proposal}

\end{document}
